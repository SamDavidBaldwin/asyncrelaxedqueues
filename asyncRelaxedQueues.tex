\documentclass[a4paper,USenglish]{lipics-v2021} % TODO: Re-add anonymous tag

\usepackage{algorithm,algpseudocode}

\bibliographystyle{plainurl}% the mandatory bibstyle

\title{Relaxation for Efficient Asynchronous Queues}

\titlerunning{Asynchronous Relaxed Queues}

\author{Samuel Baldwin}{Bucknell University, USA}{}{https://orcid.org/0000-0002-1825-0097}{}

\author{Cole Hausman}{Bucknell University, USA}{}{[orcid]}{}

\author{Mohamed Bakr}{Bucknell University, USA}{}{ORCID}{}

\author{Edward Talmage}{Bucknell Univserity, USA}{elt006@bucknell.edu}{ORCID}{}

\authorrunning{S. Baldwin, C. Hausman, M. Bakr, E. Talmage} %TODO mandatory. First: Use abbreviated first/middle names. Second (only in severe cases): Use first author plus 'et al.'

\Copyright{Samuel Baldwin, Cole Hausman, Mohamed Bakr, Edward Talmage} 

\ccsdesc[100]{\textcolor{red}{Replace ccsdesc macro with valid one}} %TODO mandatory: Please choose ACM 2012 classifications from https://dl.acm.org/ccs/ccs_flat.cfm 

\keywords{Distributed Data Structures, Asynchronous Algorithms, Relaxed Data Types} %TODO mandatory; please add comma-separated list of keywords

\funding{Funding provided by Bucknell University}

%\acknowledgements{I want to thank \dots}%optional

%Editor-only macros:: begin (do not touch as author)%%%%%%%%%%%%%%%%%%%%%%%%%%%%%%%%%%
\EventEditors{John Q. Open and Joan R. Access}
\EventNoEds{2}
\EventLongTitle{42nd Conference on Very Important Topics (CVIT 2016)}
\EventShortTitle{CVIT 2016}
\EventAcronym{CVIT}
\EventYear{2016}
\EventDate{December 24--27, 2016}
\EventLocation{Little Whinging, United Kingdom}
\EventLogo{}
\SeriesVolume{42}
\ArticleNo{23}
%%%%%%%%%%%%%%%%%%%%%%%%%%%%%%%%%%%%%%%%%%%%%%%%%%%%%%


\begin{document}
\maketitle

%%%%%%%%%%%%%%%
%%%%%%%%%%%%%%%
\section{Introduction}

This paper proposes a new algorithm for implementing a standard FIFO queue in a fully asynchronous message passing mode. To our knowledge we have not encountered any previous work that suggested a queue algorithm for this specific model. Our algorithm utilizes vector clock timestamps that allow individual processes to hold some view of the time steps of other processes have taken at points of communication. Using this technique to timestamp invocations of the two operations of the queue, then linearizing all of the queue invocations based on the ascending lexicographic order of these timestmps creates a valid permutation that meets the specifications of a FIFO Queue.  The goal of establishing this queue method is to design a system with full replication. There are existing systems that handle the asynchronous queue (server client model), but are incapable of replication. With replication, and (INSERT THE CITATION TO THE PAPER HERE) we hope that this queue algorithm can be used in fault tolerant systems in the future.

%%%%%%%%%%%%%%%
\subsection{Related Work}

%%%%%%%%%%%%%%%
%%%%%%%%%%%%%%%
\section{Model and Definitions}

%%%%%%%%%%%%%%%
\subsection{Asynchronous System Model}
We assume a fully asynchronous message passing model, with a set of $n$ processes $\Pi = [p_0, \dots , p_{n-1}]$ modeled as state machines. These state machines are time-free, meaning that their output is only described through their input and state transitions without any specific time bound. All inter-process communication is assumed to be reliable i.e any message that is sent will always be received at it’s destination process.  Additionally, communication channels between processes are considered to be treated as FIFO, in that any message sent from process $A$ to process $B$ will be processed prior to any other message from the same pair sent later in an execution. This can be accomplished by assuming that each process has an incoming buffer for each other process, and that inter-process messages are marked numerically, and out of order messages are stored in the buffer until they can be retrieved in order.  We also assume no processes crash, and none of the processes have access to a hardware clock.

%%%%%%%%%%%%%%%
\subsection{Queue Definitions}

We introduce the definition for a standard FIFO Queue abstract data type. We will use the special character $\bot$ to represent an empty queue.

\begin{definition} A \emph{Queue} over a set of values V is a data type with two operations:
  \begin{itemize}
  \item $Enqueue(val,-)$ adds the value X to the queue
  \item $Dequeue(-, val)$ returns the oldest value in the queue.
  \end{itemize}

  A sequence of queue operations is legal iff it satisfies the following conditions.  The empty sequence is a legal sequence.  Each Enqueue value is unique. Once a value has been enqueued to the queue once, it can not be enqueued again.  If $\rho$ is a legal sequence of operation instances, then $\rho \cdot Dequeue(-, val), val \neq \bot$ is legal iff Enqueue(val, -) is the first $Enqueue$ instance in $\rho$ that does not already have a matching $Dequeue(-, val)$ in $\rho$. Furthermore, $\rho \cdot Dequeue(-, \bot)$ is legal iff every $Enqueue(val, -)$ in $\rho$ has a matching $Dequeue(-,val)$ in $\rho$.
\end{definition}

%%%%%%%%%%%%%%%
%%%%%%%%%%%%%%%
\section{Asynchronous FIFO Queues}

%%%%%%%%%%%%%%%
\subsection{Description}

Each process stores a vector clock timestamp that holds its local view of the clocks at all other processes. Thus the vector clock timestamp at vi, dictating the vector clock view of process i is an array of size n (number of processes in this system) that is initially 0 at all indices. At the point when a process starts an $Enqueue()$ or $Dequeue()$ invocation, that process will increment their own index in the local clock timestamp, which marks a step or clock tick.  Processes also update the local view of the vector clocks when they receive a message containing a timestamp from another process. This occurs each Enqueue and Dequeue invocation and response, as as a part of the payload of each of these messages, the timestamp at the point in which it was invoked is included. This results in processes regularly updating the local views of their clocks, resulting in processes being as updated as possible. To update the local vector clock, values at corresponding indexes in the local clock and message clock are compared, and the larger of the two values is set to the local view. By adjusting each of these indices, this guarantees that a local clock will have the largest of the two indices at all indices in the local clock, and therefore has the most advanced possible clock at that point in the execution.  For any two vector timestamps $v_ii$ and $v_j$ , such that $v_i \prec  v_j$, states that $v_i$ is a strictly smaller timestamp than $v_j$, if $v_i[x] < v_j[x],\forall x \in [0, \dots, n-1]$. If this is not true, there is some index k, the first element that is different between the two vector timestamps $v_i$ and $v_j$ i.e. for $x = 0$ to $k-1, v_i[k] < v_j[x]$ but at index $k$, $v_i[k] \geq v_j [k]$. Then we say that $v_i$ is lexicographically smaller than $v_j$, $v_i << v_j$ if $v_i[i] < v_j[i]$. Notice that $v_i \prec v_j$ implies that $v_i << v_j$ but not the opposite.

Within each process there is a local version of an augmented minimum priority queue keyed on lexicographic timestamp order. This priority queue can perform three operations: $insert(value, v_{clock})$, $get(position)$, $remove(value)$. The insert operation adds the value to the queue based on the vclock as a priority. The $get(position)$ function allows the user to peek into the element at the passed position without removing a value from the queue. The $remove(value)$ function removes the specific value passed to it from the queue which can be at any location in the queue. Ordering elements in FIFO order is not a straight-faced task in a distributed setting since defining which invocation happened first is not as clear as in a linear setup. Consequently using a priority queue keyed on lexicographic timestamp order allows us to ensure some order locally to guarantee FIFO consistency to the user once linearized.

%%%%%%%%%%
\subsubsection{Confirmation Lists}
The main structure of this algorithm it to utilize a structure we will define called a \emph{Confirmation List} to track the responses for a given dequeue locally for each process. Upon receiving a dequeue request message, a given process will either declare that dequeue ”safe” or ”unsafe” if that process is or isn’t in the active process of dequeuing an element. If a process will declare that message safe, it sends a message confirming that to the process that invoked the dequeue, and if it will declare the message unsafe, it sends a message stating that to all processes in the system. Both safe and unsafe messages will be marked with the invoking process, the responding process, and the invoking dequeue vector clock.

Upon hearing about a dequeue request, which are sent globally, a process will instantiate a Confirmation List in it’s local memory, indexed by the timestamp of the invoking processes dequeue. For the dequeue invoking process, the confirmation list will fill by recieving messages from the other processes in the system with a corresponding timestamp. A safe message will be marked as a 1 in the corresponding index of the process in the Confirmation List and an unsafe will be marked as a 2. Once the Confirmation List has no undefined indices, the number of 2’s within the list will be counted, and the queue will be accessed at that index, and the element removed.

For processes that are not the invoking process, only the unsafe messages are received, and marked as 2 in the local Confirmation Lists. To fill the rest of the list, messages with timestamps strictly greater than the invoking timestamp will be treated as implicit ”safe” messages, as there cannot be an unsafe message that has not been received from that process. Thus, with the same number of unsafe messages, the non-invoking processes will naturally come to the same conclusion and remove the same element as the invoking processes.  Confirmation lists, like enqueue requests are sorted by timestamp order, as when they are created, the invoking dequeue timestamp is included. The general structure of a of conf list is as follows [[Flags for responses from a given process][Invoking Process ID][Timestamp of invocation]]


%%%%%%%%%%%%%%%
\subsection{Algorithm}

%%%%%%%%%%%%%%%
\subsection{Correctness}

\begin{lemma}
  Every invocation has a matching response.
\end{lemma}

\begin{proof}
Based on teh specifications of the algorithm, if there is an $Enqueue$, the request ffor that is sent globally and responded to.  In teh case of a $Dequeeu$, the request is sent globally, and the response for that $Dequeue$ iss also sent globally.  In all cases of requests being sent, there is a delay of $2d$ , where $d$ is the maximum message delay.
\end{proof}

\begin{lemma}
  The local views of the queues are equivalent at the time of $Dequeue$.
\end{lemma}

\begin{proof}
  For this algorithm, we will consider the entire history of the execution, such that all elements that were at one point in the queue are considered a part of the prefix of the queue. Once an element is dequeued, that element remains a part of the prefix, but no longer can be accessed, and is marked unavailable for dequeues.
  
While processes may have different complete histories for the histories of their enqueues at some point in time, given they may be in different local states with respect to active enqueues that they may have not recieved, the prefix of all local views up to the timestamp of the local execution of a dequeue will be the same. This is due to the fact that there is an agreed upon order for sorting the enqueues, and as such all processes will agree on the order. Furthermore, at the time of a dequeue, we can confirm that a given process will have heard of all prior enqueues. This is due to the fact that in order to locally execute a dequeue, a given process will have to have heard all messages prior to that dequeue, via the Confirmation List structure. The nature of the communication between processes being FIFO results in the knowledge that there cannot have been earlier messages that have gone unrelieved if a later message has been processed and therefore, if a process is actively completing a dequeue invocation, at the point of that invocation every message prior has been received. Thus, we can formally state that at such a point, the prefix of all messages and queue states are equal across all processes.
\end{proof}

\begin{lemma}
In order to dequeue an element, there must be a corresponding $Enqueue$.
\end{lemma}

\begin{proof}
Proof by contradiction. Assume that some process invokes a dequeue, receives responses and locally executes, removing some element that was not enqueued.  That element, by the algorithm definition, must be held within the local view of the queue. This is because dequeues are accessed via the index provided by the unsafes. Therefore, the element that was dequeued by the process must exist in the local queue. The only way that can occur if the element was enqueued.  Additionally, when removing an element from the queue, it is important to not remove elements with a larger timestamp that the dequeue. In the case in which there is a dequeue request and a timestamp where $dqts < eqts$, and the dequeue executes locally, given we cannot linearize the enq before the deq, we must return $\bot$ and not the element. This behavior is defined in the algorithm (IM LYING ITS NOT WE NEED TO ADD IT :D), and allows us to establish a specified order.
\end{proof}

\begin{lemma}
Elements that are dequeued are unique. (No double dequeue.)
\end{lemma}

\begin{proof}
Given that at the time of execution we have proved that the head of the list of enqueues, combined with the algorithm definition guaranteeing that the local conf list of all processes contains the same information, we can state that all processes at the time of local execution will act on the same information to select an index, and therefore will select the same index. Given that, we can state that they will remove the same element from the queue, regardless of the time of execution locally.
\end{proof}

\begin{theorem}
  Algorithm~\ref{alg:fifo} correctly implements a FIFO queue.
\end{theorem}

\begin{proof}
Based on the information of the lemmas, we can conclude that the algorithm definition for the queue follows the specifications for the ADT. Enqeues have corresponding dequeues, no double dequeues, we can linearize by timestamp such that the first element in the queue is always removed first.
\end{proof}

  

%%%%%%%%%%%%%%%
\subsection{Complexity}

%%%%%%%%%%%%%%%
%%%%%%%%%%%%%%%
\section{Asynchronous Out-of-Order Queues}

%%%%%%%%%%%%%%%
\subsection{Description}

%%%%%%%%%%%%%%%
\subsection{Algorithm}

\begin{algorithm}
  \caption{Code for each process $p_i$ to implement a Queue with out-of-order k-relaxed \textit{Dequeue}, where $k \geq n$ and $l = [k/n]$}
  \begin{algorithmic}[1]
    \Function{Enqueue}{$val$}
      \State $EnqCount = 0$
      \State $incrementTS(v_i)$
      \State $Enq_{ts} = v_i$
      \State send $(EnqReq, val, i, Enq_{ts})$ to all processes
    \EndFunction

    \Function{Receive}{$EnqReq, val, j, Enq_{ts}$} from $p_j$
      \State $updateTS(v_i, Enq_{ts})$
      \State $lQueue.insertByTS$($val, j, Enq_{ts}$)
      \For{$confirmationList$ in $PendingDequeues$}
        \If {$confirmationList.ts \prec Enq_{ts}$}
          \State $confirmationList.responses[j] = Safe$
        \EndIf
      \EndFor

      \State send $(EnqAck, i)$ to $p_j$
    \EndFunction

    \Function{Receive}{$EnqAck, j$ from $p_j$}
      \State $EnqCount += 1$
      \If {$EnqCount == n$} 
        \Return $EnqResponse$
      \EndIf
    \EndFunction

  \end{algorithmic}
\end{algorithm}

\begin{algorithm}
  \caption{Continued, part 2}
  \begin{algorithmic}[1]
    \Function{Dequeue}{}
      \State $incrementTS(v_i)$
      \State let $Deq_{ts} = v_i$
      \If {$localQueue.peekByLabel(p_{i}) \neq \bot$}
        \State let $ret = localQueue.deqByLabel(p_i)$
        \State send $(Deq_f, ret, i, Deq_{ts})$ to all processes
      \Else
        \State send $(Deq_s, null, i, Deq_{ts})$ to all processes
      \EndIf
    \EndFunction

    \Function{Receive}{($op, val, j, Deq_{ts})$ from $p_j$}
      \State $updateTs(v_i, Deq_{ts})$
      \State let $p_{invoker} = p_j$
      \If{$Deq_{ts}$ is not in $PendingDequeues$}
      \State $PendingDequeues.insertByTs(createList(Deq_{ts}, p_{invoker}$))
      \EndIf
      \If{$Deq_{ts} \neq 0$ and $Deq_{ts} \prec v_i$}
        \State let $safetyFlag = Unsafe$
      \Else
        \State let $safetyFlag = Safe$
      \EndIf
      \State send $(op, val, safetyFlag, Deq_{ts}, i, p_{invoker})$ to all processes
    \EndFunction

  \end{algorithmic}
\end{algorithm}

\begin{algorithm}
  \caption{Continued, part 3}
  \begin{algorithmic}[1]
    \Function{Receive}{$(op, val, safetyFlag, Deq_{ts}, i, p_{invoker})$ from $p_j$}
      \If{$Deq_{ts}$ not in $PendingDequeues$}
        \State $PendingDequeues.insertByTs(createList(Deq_{ts}, p_{invoker}))$
      \EndIf
      \State $propagateEarlierResponses(PendingDequeues)$ 
      \For{($index, confirmationList)$ in $PendingDequeues$}
      \If{$confirmationList.ts == Deq_{ts}$}
          \State $confirmationList.responses[j] = safetyFlag$
        \EndIf 
        \If{$confirmationList.full()$ and not $confirmationList.handled$}
          \State $confirmationList.handled = True$ 
          \If {$op == Deq_f$ and $p_i \neq p_{invoker}$}
            \State $lQueue.remove(val)$
          \EndIf 
          \State $updateUnsafes(PendingDequeues, index)$
          \If {$op == Deq_f$}
            \State \Return
          \EndIf

          \If {$localQueue.peekByLabel(p_{invoker}) \neq \bot$}
          \State let $ret = lQueue.deqByLabel(p_{invoker})$
          \Else 
            \State let $pos$ = Number of $Unsafe$ flags in $confirmationList.responses$
            \State let $ret = lQueue.deqByIndex(pos)$
          \EndIf
          \State $labelElements(p_{invoker})$
          \State $updateUnsafes(PendingDequeues, \textit{index})$
          \If{$p_i == p_{invoker}$}
            \State \Return $ret$
          \EndIf
        \EndIf 
      \EndFor
    \EndFunction
  \end{algorithmic}
\end{algorithm}


%%%%%%%%%%%%%%%
\subsection{Correctness}

In the case where $k > n$, since the number of messages required to have a successful dequeue in bounds is less than the number of permitted dequeues ooo, dequeues can be instant. This information is based off of PUT THE PAPER NAME HERE which utilizes prior labeling of elements within the queue structure with a process that is allowed to instantly dequeue it.  Given each element is distinct and can only be dequeued once, the labeling must be distinct, in that each process agrees on the labelling that a given value in the queue receives. Whereas in PAPER NAME AGAIN this is done via a process of synchronization of clocks, we will be adjusting the labelling to occur at points when we can guarantee that the view of distinct processes agree, at the point of dequeueing. As we have discussed, at the point in which a process dequeues, the information prior to that dequeue, the prefix to the execution $\rho$ to that point will be the same for all processes. Given that all processes will agree on a select set of information at the same relative point in their executions, we can treat this as a sort of synchronization, and make decisions as to the labelling of elements at this point.

For this relaxation to still follow the established queue definition, it must only return elements once. Therefore, we must prove that a) Each process dequeues each element only once and b) that elements removed are within the first $n$ elements within the queue.  Given that the labeling is agreed upon by every process, and done by calculations with an invariant, we can state that each process will agree on the elements that are ”fast dequeued” with respect to which process is allowed to ”fast dequeue” them. Given that processes agree which elements are ”reserved” for a fast dequeue by another process, we can furthermore state that a process will not slow dequeue an element that has been marked.

Thus, elements can not be dequeued by processes that they have not been marked for, so we can guarantee that they will ONLY be dequeued by the assigned process.  By thje definition of the algorithm on line GIVE ME THE LINE NUMBER HERE COLE we can also state that the second criterion is satisfied, as at a point where a given process will attempt to fast dequeue, the position of the element that is attempting to be fast dequeued is checked, and the process is only allowed to execute the operation in the case where the element is within the legal bounds for dequeueing.


%%%%%%%%%%
\subsubsection{Less Relaxation than Number of Processes}
In the case in which the X of out of order is less than the number of processes,there isn’t an order of magnitude benefit to relaxation. The benefit of relaxation is that rather than forcing a wait of 2 times the maximum message delay, the wait can be cut to 2 times the (n-x+1)th slowest message delay. In a close connection system, the benefit is minimal as the difference between the max message delay and the (n-x+1)th is minimal, but in an actualized system with nonstandard message delays, this could have some substantial benefits.

%%%%%%%%%%%%%%%
\subsection{Complexity}

%%%%%%%%%%%%%%%
%%%%%%%%%%%%%%%
\section{Conclusion}

\bibliography{refs.bib}

\end{document}
o

